\chapter*{Introduction}
De nos jours, le premier comme le dernier des imbéciles peut accéder à l’édition et pour les plus connus d’entre eux accéder aux médias de masse. La notoriété ouvre bien des portes, surtout celles des studios qui mènent vers de confortables canapés rouges. Une fois bien installé l’invité va passer tout son dimanche après midi à se faire cirer les chaussures par un présentateur aussi indéboulonnable que centenaire. Le tout sous l’œil avide et à moitié éteint de la fameuse ménagère de plus de cinquante ans. \\
Les rayons « culturel » des grandes surfaces croulent littéralement sous la masse des essais, autobiographies et hagiographies des illustres stars en tout genre. A croire qu’on puisse acheter les livres au kilo plutôt qu’à la pièce. Sachant qu’on trouve déjà des offres pour l’achat en lot, ce ne serait que la suite logique de cette évolution mercantile. Littérature et lessive, même combat. \\
Il fut sans doute une époque, que je n’ai pas connu malgré mon âge assez avancé, ou réussir à faire publier son manuscrit était un gage de talent. Peut-être quand ces temps-là : la pâte à papier était beaucoup plus chère et les lecteurs plus difficiles. Peut-être que cette époque, tant louée, remonte au temps des moines copistes bien avant l’invention de l’imprimerie. Peut-être que cette époque si merveilleuse n’a jamais existé ailleurs que dans l’imagination de certains utopistes un peu niais.\\

Loin de moi l’idée de me prétendre meilleur ou pire que le dernier des imbéciles capable de pondre un ouvrage. Nul doute que j’occupe une place intermédiaire dans l’échelle des crétins et que je suis talonné de prêt par deux idiots en tout points mes semblables. \\

J’ai décidé de me lancer dans un projet tout particulier. Je m’en vais investir les réseaux sociaux comme l’écrasante majorité de mes contemporains. Tout comme eux, je ne vais pas chercher la vérité avec une grande majuscule, mais au contraire je vais y aller de mes petits arrangements, mensonges et approximations. Là où je vais m’inscrire en négatif par rapport à la tendance majoritaire c’est qu’au lieu d’enjoliver l’ordinaire : je vais forcer le trait jusqu’au ridicule et à l’absurde. Ma vie n’est, sans doute, pas plus ennuyeuse que la moyenne, mais je vais la rendre plus plate, plus morne encore en m’attardant sur toutes les futilités qui échappent au commun. Je m’en vais m’étaler de tout mon long dans les pires lieux communs, les aphorismes les plus éculés, les digressions les plus interminables. \\
A mon tour d’apporter ma pierre à la médiocrité ambiante et de laisser mon empreinte sur la fange de ce siècle. Mesdames, messieurs laissez-moi vous présenter mon œuvre : ma petite vie vue par le trou de la serrure et relayée à la face du monde par les génies du numérique. Toute ma misérable existence relayée à la multitude et sauvegardée à jamais dans le silicium pour les siècles des siècles. 

\footnote{Il n’aura pas échappé au lecteur averti que j’emprunte le titre de mon tapuscrit à la série allemande Mein Leben und Ich.}
