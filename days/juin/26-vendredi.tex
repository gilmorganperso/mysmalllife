\section*{Vendredi 26}
\subsection*{Pain au raisins}
Je n’ai pas mis les pieds dans une boulangerie de la journée pourtant, allez savoir pourquoi, j’ai une folle envie de pain au chocolat. Sans s final, une seule de ces viennoiseries comblera aisément mon envie. Je n’ai pour autant pas fin, c’est juste une envie qui m’est venue d’un coup. \\
Je précise que je ne suis pas enceinte. Même si je le voulais cela me serait rigoureusement impossible. C’est simplement que j’ai un peu trop forcé sur les abdos et ça dépasse du t-shirt et cela n’a rien à voir. \\

Soyons précis, ce n’est pas exactement d’un pain au chocolat que j’ai envie. Sinon j’aurais eu vite fait de prendre ma voiture, d’enfoncer la porte d’une boulangerie et de demander d’une voix forte et assurée un pain au chocolat. J’en serais parfaitement capable, je l’ai d’ailleurs fait de nombreuses fois par le passé. Oui mais, sans craindre de me répéter, ce n’est pas exactement d’un pain au chocolat (ce que les ignares nomment chocolatine) que j’ai envie mais d’une variante. En matière de viennoiserie, il y a certes quelques classiques qu’ils convient de laisser en l’état et ces exceptions de côté, rien doit limiter la créativité des plieurs de croissant. Rien si ce n’est le bon goût bien sur. \\

Prenons par exemple le fameux pain aux raisins que l’on nomme aussi : escargot, couque aux raisins, couque escargot, couque suisse, pain russe, pain suisse, schneck ou que sais-je encore. Il n’existe à ce jour aucun organisme chargé de veiller à l’uniformisation des appellations pâtissières au sein de la francophonie. Pour une fois que les académiciens pourraient se rendre utile, il faut croire qu’ils sont trop occupés à savoir combien de n il faut mettre a zigounette pour s’occuper des choses pratiques. \\
Tout est-il que peu importe le nom qu’on lui donne cette pâtisserie ressemble quasiment toujours à la même chose. Une pâte feuilletée roulée en spirale garnie d’une crème pâtissière et de raisins secs. C’est un classique indémodable. Imaginons maintenant que l’on substitue les raisins secs pour les remplacer par des pépites de chocolat. Vous y êtes. On ne pourrait appeler cette création : pain au chocolat (l’appellation est déjà prise) pas plus que pain aux raisins (ce serait mensonger). La locution : pain aux raisins sans les raisins mais avec des pépites de chocolat à la place, est bien trop longue pour être pratique. \\

Une chose est sure, c’est que cette petite merveille existe. J’ai eu le plaisir d’y goûter plusieurs fois dans un charmant établissement en bord de Sarthe. Pardonnez-moi l’expression mais c’est une tuerie ce gâteau. Je n’arrive toujours pas à m’expliquer pourquoi on en trouve pas un peu partout ? 
