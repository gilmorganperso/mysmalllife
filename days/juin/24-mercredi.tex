\section*{Mercredi 24}
\subsection{shampoo}

Je suis assez surpris et un peu déçu en même temps par les efforts que font certaines marques pour les enfants. Comprenons-nous, ce n’est pas les efforts que les équipes marketing qui me gênent. Si on met de coté le bien fondé de leur démarche purement mercantile qui consiste à transformer les petites têtes blondes en consommateurs de masse : il faut avouer qu’ils sont plutôt doués. Le mal qu’ils font : ils le font bien. \\
J’étais dans ma douche, et je vous l’avoue sans honte, j’ai une nouvelle fois volé le shampoing de mon fiston. En tant que mâle alpha de la meute, j’ai parfaitement le droit de préemption sur tout ce qui me tombe sous la main. Ou presque. \\

Comparons un instant les deux bouteilles de produit. D’un côté nous avons une bouteille tout à fait quelconque et banale avec un logo fade représentant une noix de coco et une fleur de vanille. Un packaging que l’on aurait tout aussi bien pu accoler sur un produit alimentaire, un nappage pour gâteaux industriels par exemple. De l’autre nous avons un flacon en forme de manchot empereur rouge vif avec sur l’étiquette : deux extra terrestres hilares, dotés d’une paire et demi de globe oculaires chacun, la tête recouverte d’une épaisse charlotte de bulles savonneuses. Là au moins, le doute n’est pas permis : la fonction du produit est on ne peut plus explicite. De plus, il n’y a qu’a se fier aux deux immenses sourires des bestioles pour comprendre qu’il s’agit d’un produit de bonne qualité. \\
Mais ce n’est pas tout, s’il n’y avait que l’emballage pour faire la différence, les choses seraient sans doute plus acceptables. Bien sur ce n’est pas du tout le cas. Mon fiston dispose d’un shampoing parfumé au fruit du dragon (hylocereus undatus de son petit nom latin). Je ne sais pas qui est l’imbécile de première classe qui a décrété l’usage exclusif au moins de vingt ans de ce fruit. Peut-être est-ce un scribouillard aigri d’une commission européenne obscure en charge de ce genre de problématique. Une chose est sure, ce n’est pas en procédant ainsi qu’il va s’attirer ma sympathie.\\
Au rayon « adulte » il est impossible de trouver ce type de parfum, ne cherchez pas c’est une peine perdue.\\

Je me doute qu’il y a derrière ce choix une logique, du moins je l’espère, mais j’ai beaucoup de mal à me figurer le pourquoi du comment. Sans doute qu’une personne ayant ses entrées dans le monde du marketing pourrait éclairer ma lanterne.

