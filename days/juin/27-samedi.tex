\section*{Samedi 27}
\subsection{L’Europe des yaourts}
L’Europe des yaourts n’existe pas et sans doute qu’elle ne se fera jamais. Malgré tout les beaux discours qui fleurissent sur la concorde entre les pays de la zone euro, il reste encore beaucoup de chemin à parcourir. Sans vouloir apporter de l’eau aux moulins des eurosceptiques de tout poil, il faut bien avouer que l’Europe des yaourts n’existe pas. \\
Je ne veux pas parler des quotas laitiers ou des diverses réglementations en vigueur dans le domaine agricole. Connaissant la compétence des diverses institutions technocratiques qui règnent sur Bruxelles, je ne suis pas certains qu’ils se soient entendus sur la définition de vache. Tout le monde sait bien que vache est un nom vernaculaire donnée à la femelle du Bos taurus, je ne m’explique pas qu’il y ait besoin d’épiloguer plus longtemps sur le sujet. \\

Revenons à nos yaourts. Pour illustrer mon propos et aiguiller mon raisonnement je vais me restreindre à un exemple concret. Prenons un yaourt grec sur un coulis de fruit, je ne citerais pas de marque sauf si celle-ci me fait une proposition commerciale suffisamment alléchante pour faire plier ma déontologie. A toute fin utile, messieurs les commerciaux : deux douzaines de ces fameux desserts feront amplement l’affaire. \\
Ce yaourt grec est vendu en France et aussi en Espagne, sans doute même en Grèce mais je n’en ai pas la certitude. Pour couper court à toute polémique, sachez que ce yaourt aurait tout aussi bien pu être bulgare que cela n’aurait pas changé le fondement de mes propos. Or il se trouve que ce yaourt est grec, ce n’est ni une qualité ni un défaut : c’est juste un état de fait. Le point important à souligner c’est qu’il n’est ni français ni espagnol. \\

En toute logique on aurait pu croire que ce yaourt soit identique d’un côté comme de l’autre des Pyrénées. Et bien non, absolument pas. Ces yaourts pourtant en tout points identiques en apparence n’ont pas le même goût, ni la même onctuosité. Ils sont largement meilleurs de l’autre côté de la frontière et c’est bien dommage, de mon point de vue s’entend car je ne peux pas faire plusieurs centaines de kilomètres pour me ravitailler. \\

Cette inégalité est tout à fait insupportable, c’est une insulte à la concorde des peuples. Rien ne justifie qu’un yaourt grec soit meilleur dans la péninsule ibérique que dans la vallée du Loir. Il n’y a aucune raison, à ma connaissance, historique qui justifierait un traitement de faveur envers les Espagnols de la part des Grecs. \\
Messieurs les industriels corrigez vite cette erreur avant de déclencher un conflit qui serait dommageable autant à vous qu’au reste des peuples.
