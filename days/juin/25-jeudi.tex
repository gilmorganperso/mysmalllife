\section*{Jeudi 25}
\subsection*{Pourquoi la Bretagne ?}
Je repose ma question avec un peu plus de mots pour la rendre plus aisément compréhensible par le plus grand nombre. \\
Pourquoi avez-vous choisi la Bretagne comme destination de vacances ? Certes c’est très beau, et ce n’est certainement pas moi qui vous dirais le contraire. C’est juste que dans l’imaginaire populaire français : cette belle région rime avec précipitations. L’image d’Epinal représente toujours l’Armorique sous un déluge avec quelques autochtones en ciré jaune en arrière fond. Combien d’entre nous ce sont fait, gentiment, chambré à leur retour de congés par leurs collègues. Tu es bronzé, tu n’as pas passé tes vacances en Bretagne finalement. Et bien si justement. On va se forcer à rire de cette blague aussi peu drôle qu’elle est éculée. Dans la vie, on ne choisit pas toujours ses collègues. Hélas. \\

S’il est de notoriété publique que la Bretagne est une région aussi arrosée qu’un départ à la retraite, pourquoi donc vouloir y gâcher une grande part de ses congés payés. L’être humain n’en est pas à une contradiction près, mais là tout de même je m’interroge. Serait-ce pour tenter d’expier une faute ? Mais si oui, laquelle ? Ou est-ce pour soulager un penchant masochiste inavoué. \\
Aucune de ses deux explications ne saurait me convaincre véritablement. Le jour où les Français chercheront le repentir pour toutes leurs fautes : il y aura des embouteillages de pénitents sur toutes les rocades. De même, si le masochisme était aussi répandu dans la population : on verrait des cravaches en présentation au télé-achat. Je n’ai jamais vu Pierre Bellemare avec un martinet à la main. \\

L’explication, si elle existe, est forcément ailleurs. \\

Cette année, plus encore que les précédentes, le flux migratoire estival en direction de la Bretagne est immense. C’est à croire que le reste de l’hexagone se vide pour chercher à s’entasser dans quatre petits départements. Elle doit bien être vide la Côte d’Azure au mois d’août. Quand je pense à toute cette horde qui va débarquer sur les côtes en quête d’un authentique factice. Tous ces pauvres imbéciles qui s’en vont s’extasier sur les mouettes avec leur beurre sucre dégoulinante dans la main et leur sac à dos Bécassine en bandoulière : j’hésite entre rire et pleurer. N’avez-vous pas un ailleurs à polluer de votre présence ? Un ailleurs de préférence loin, voir très loin de moi. \\
Vous qui toute l’année avez raillé le climat breton, si vous ne vous sentez pas ridicule avec vos méduses en plastique au pied et vos bobs du tour de France défraîchi sur le crane, ne pourriez-vous pas faire l’effort de prendre moins de place. \\
Oui, moins de place. Juste vous poussez un peu pour laisser de l’air et un coin de paysage a ceux qui savent l’apprécier à sa juste valeur. Ou simplement vous faire plus discret pour que j’en vienne à oublier votre présence. \\

D’avance, merci.

\subsection*{vampyroteuthis infernalis vous avez dit vampyroteuthis infernalis ?}
Je suis tombé sur un documentaire particulièrement intéressant sur le vampire des abysses (Vampyroteuthis infernalis). Derrière cette appellation digne d’un mauvais film de série Z à petit budget se cache un animal sans doute assez inoffensif. La bestiole en question habitant les grands fonds sous-marins, il serait difficile de la croiser au détour d’une baignade. \\
Reconnaissons tout de même qu’elle ne pourra pas gagner un concours de beauté du genre animal. Son physique n’est pas très avenant, mais nous sommes encore bien loin des canons du blobfish. Toutefois ce ne sont pas les considérations artistiques qui m’ont le plus interpellés. \\
Cet animal occupe un ordre et une famille (au sens taxonomique) à lui tout seul.   Après l’avoir classé à tort dans la famille des poulpes puis des calamars, justice lui est enfin rendu. C’est sans aucun doute dans un souci de se racheter de leurs fautes que les zoologues lui ont aménagé une branche rien que pour lui. À croire que les erreurs de jugement ne sont pas l’apanage unique de la justice. Est-ce pour autant qu’il faille s’en réjouir, c’est une tout autre histoire.
