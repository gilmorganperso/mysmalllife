\section*{Samedi 18}
\subsection*{Le dernier boulaneigier}

La boule à neige, je ne suis pas certain que tout le monde, surtout les plus jeunes, saisisse vraiment de quoi il en retourne. Alors pour ceux du fond qui n’ont pas encore tout suivi, voici une petite piqûre de rappel. Une boule à neige est un dôme généralement en plastique rempli d’eau et quelques flocons avec une photo glissée à l’intérieur. Le jeu consiste à retourner ou à secouer l’objet afin de donner l’impression que la neige tombe. \\
Il fut un temps ou ce genre de bidule était très tendance. Toutes les stations balnéaires proposaient des boules à neige aux touristes de passage. Sitôt ramenés de voyage, les bibelots finissaient leur vie à prendre la poussière en haut d’une étagère.

C’était avant, avant l’arrivée du numérique entre autres choses. De nos jours, plus personne ne ramène une boule à neige dans ses bagages. Personne n’irait exhiber fièrement sa collection lors d’un repas de famille. Les boules à neige ont fini par prendre le même chemin que les diapositives : direction le cimetière des vieilleries encombrantes, les cartons du grenier. \\
Il doit bien exister une application pour faire la même chose. Un truc stupide qui ajoute de gros pixels blanc sur une photo quand on secoue son téléphone. Il existe forcément un imbécile sur la planète pour en avoir eu l’idée avant moi.
Savoir qu’il existe plus con que moi, me mets toujours en joie. Il m’en faut peu pour être heureux.

Saviez-vous qu’avant d’être fabriquées à la chaîne à des cadences infernales : elles étaient façonnées à la main par des artisans ?

La plus ancienne boule à neige attestée aurait été l’œuvre d’un apprenti de chez Fabergé. Inspiré par les œufs de pâques de son maître il aurait fabriqué une boule en verre renfermant un kremlin miniature. En secouant l’œuvre on voyait la neige tomber. Toujours selon l’histoire : il aurait destiné ce cadeau à une cantatrice un peu volage pour qu’elle pense à lui lors de sa tournée parisienne. Les mêmes historiens s’accordent à dire qu’elle aurait fait une bonne carrière dans un registre différent et que toute l’amirauté de l’époque aurait loué ses talents.


