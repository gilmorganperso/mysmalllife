\section*{Jeudi 9}
\subsection*{le pchit de la mort}
L’idée m’est venue en regardant une série sur Netflix, le titre de cette série n’a pas vraiment d’importance. D’ailleurs, je ne me souviens même plus c’est pour dire. Bref je regardais cette série dans les meilleures conditions possibles pour l’occasion. C’est-à-dire : inconfortablement installé sur ce qui fera passé les bancs des cantines scolaires pour des sofas italiens dans la pénombre des ampoules en fin de vie. L’ambiance angoissante et pesante de la série était bien mis en valeur par le décor de la pièce, il ne manquait qu’un tout petit plus pour que cela puisse atteindre la perfection.\\
Le petit plus, je l’ai trouvé. Il s’agit à l’origine d’un parfum d’ambiance à piles, le genre de pseudo vase en plastique que l’on pose sur un meuble et qui déclenche un jet de déo à chaque mouvement et, ou à intervalle régulier. L’engin en question était aussi vétuste que le reste du mobilier. Au lieu de se contenter d’un petit pchit discret à la limite de l’audible, il faisait un craquement sinistre. Imaginons un peu que l’on perfectionne l’engin afin d’en faire un générateur de jump-scare. Un petit mécanisme qui émettrait des bruits sinistres de manière aléatoire ou presque dans le dos des spectateurs des films d’horreur. Combien d’entre nous ont sursauté durant une projection non pas à cause d’un effet du film mais à cause d’un simple courant d’air. Faites claquer une porte durant un visionnage de l’exorciste et vous aurez au moins un arrêt cardiaque dans l’assistance. Non, ne le faites pas ou ne dites pas que c’est moi qui vous aie soufflé l’idée, je ne veux pas de problème avec la justice.\\
Imaginez maintenant l’effet que pourrait produire un générateur de grincement de parquet et de porte qui claque. Sans vouloir me vanter, si ce n’est pas l’idée du siècle : cela y ressemble beaucoup.
