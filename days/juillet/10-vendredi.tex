\section*{Vendredi 10} 
\subsection*{Le produit à douche sans}
Si vous avez eu le courage de me suivre depuis le début, vous allez sans doute croire que je souffre d’obsessions maniaques. Sachez d’une part que je loue votre courage et d’autre part que vous n’avez pas complètement tort mais qu’il me pèse beaucoup de le reconnaître. \\
A proprement parler je ne souffre pas d’une obsession particulière pour le produit à douche, c’est juste que je passe au minimum une foi par jours sous le jet. Alors forcément à force de manipuler le même flacon, on finit par s’intéresser au contenu. Si vous mangiez tous les jours les mêmes boites de conserve, à la longue vous en connaîtriez la composition sinon par cœur mais au moins dans les grandes lignes. \\
Ce qui a attiré mon attention aujourd’hui c’est l’énorme mention « sans » barrant l’étiquette de sa typographie criarde. Il fut un temps où l’on vantait les ingrédients d’une recette, l’heure est semble-t-il à la volte complète. Le flacon affiche un énorme « sans » sous lequel s’alignent les grands absents.
En première ligne, le parabène. Tout le monde sait bien que le parabène est un parahydroxybenzoate d'alkyle, c'est-à-dire un ester résultant de la condensation de l'acide parahydroxybenzoïque avec un alcool. Est-ce là une raison suffisante pour le condamner ? Permettez-moi d’en douter. Sur ce point la communauté scientifique reste divisée. Reste le principe de précaution mais est-ce pour autant une raison valable pour se dispenser de ses bienfaits antifongiques et antibactériens ? \\
Ensuite vient la mention sans huile de palme. La question n’est pas tant de chercher à démontrer la nocivité de l’huile de palme mais surtout son utilisation dans ce genre de produit. Suis-je le seul à m’étonner qu’on puisse se huiler sous la douche. Tout le monde est libre de faire ce qu’il veut sous la douche, dans le respect des principes fondateurs de la loi, pour le reste : je ne juge pas. \\
C’est principalement la dernière ligne qui m’a laissé sur le séant : sans savon. Le savon c’est quand même le propre de l’homme. A la base : un produit à douche c’est quand même prévu pour se décrasser alors pourquoi retirer le savon et surtout par quoi le remplacer. Sauf grossière erreur de ma part, le savon est utilisé depuis des siècles et pas grand monde n’y trouvait grand-chose à y redire. Sauf peut-être les prisonniers quand ils faisaient tomber leur savonnette dans la douche. Les amateurs de Midnight Express comprendront.
Bref, quand j’additionne tous les moins de la liste des ingrédients : exit le colorant, les parfums de synthèse et les agents de texture, je m’étonne encore de trouver quelque chose au fond du flacon. Allez comprendre ! 
