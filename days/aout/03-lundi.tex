\section*{Lundi 3}
\subsection*{Archimède blues}
L’illumination, non que dis-je, le déclic m’est venu une nouvelle fois sous la douche. Archimède avait sa baignoire, j’ai ma cabine de douche. Chacun fait selon ses moyens.\\
Le flacon de produit que je tenais entre les mains indiquait fièrement la mention « non testé sur des animaux ». Je suis sorti en urgence, ruisselant encore de mousse pour me jeter sur mon téléphone et tout annuler. Avant d’aller plus loin dans les explications qui s’imposent, je vous propose de revenir en arrière et de commencer par le début. \\
Tout à commencer quand j’ai soumis l’idée de donner des hallucinogènes à une autruche. Enfin non, j’ai plutôt exprimé l’opinion que cette idée était particulièrement mauvaise et que je me félicitais de ne pas l’avoir eue. Pour de plus amples détails, je vous renvoie à la lecture de mon billet précédent. Ce qui devait arriver ne s’est pas fait attendre. J’ai reçu un message fort intéressant d’un employé de zoo qui m’a raconté avoir testé quelques expériences vaguement similaires par le passé. Par exemple, il s’est amusé à donner des laxatifs aux éléphants. Le résultat a été vraiment spectaculaire, éruptif même pour reprendre ses mots. Les pachydermes ont très, trop, bien réagis a cette médication. Peut-être s’est-il trompé dans les dosages, il m’a avoué ne pas maîtriser toutes les subtilités de la pharmacopée humaine appliquée aux animaux. Une autre fois, il a refilé des petites pilules bleues aux marmottes. L’enclos a été temporairement interdit aux visites scolaires durant presque deux semaines. Le temps que la libido de ces bestioles redescende à un minimum convenable pour un public non averti. \\
En lisant le mail de ce monsieur, j’en suis naturellement arrivé à la conclusion suivante. Il fallait absolument que je le rencontre, afin bien sur, de le dissuader de recommencer une de ses nouvelles expériences. Souvenez-vous bien les petits enfants : la drogue c’est caca. 

Après quelques verres en sa compagnie, avec quelques tendant lourdement vers moult, nous avons finalement décidé d’épargner les autruches, les émeus et les casoars pour nous rabattre sur les lémuriens.\\
Les lémuriens ressemblent à des singes, même s’ils n’appartiennent pas du tout au même ordre. Ce sont des peluches avec de grands yeux exorbités et les gestes au ralenti. En sommes les candidats idéaux pour se voir prescrire du speed.

Avant que les défenseurs des animaux ne s’emparent de leurs fourches et de leurs flambeaux : sachez que nous avons pensé avant tout au bien être de ses petits junkys en fourrure. Nous ne pouvions les laisser dans la nature toute relative du zoo après avoir ingurgité une bonne dose de stupéfiant. Ce que nous avions prévu, c’était de les extraire de leurs cages pour les plonger dans un habitat plus rassurant. Imaginez-vous en plein milieu d’animaux sauvages dont l’immense majorité d’entre eux seraient des prédateurs sanguinaires. Chaque cri de hyène résonne dans le crâne du petit lémurien comme un glas funèbre. Le plus sécurisant aurait été de le placer dans un environnement plus feutré et édulcoré. Chez ma belle mère par exemple.

On a passé une bonne partie de la soirée à planifier l’opération dans ses moindres détails. Nous devions passer prendre notre patient quelques heures après la fermeture du parc zoologique. Le patient prendrait place dans le siège enfant à l’arrière de ma voiture, l’autoradio diffusera du Charles Aznavour en sourdine. Selon Yannick [c’est un prénom d’emprunt afin de garantir l’anonymat] les lémuriens ont un faible pour ce genre de musique. J’ignorais qu’il existait un top 50 chez les lémuriens. Si le grand Charles s’était produit dans l’enceinte du zoo : il aurait fait cage comble si vous me passez l’expression. On ferait bien attention à conduire suffisamment lentement pour ne pas inquiéter notre passager mais pas trop pour ne pas attirer la maréchaussée. Une fois arrivé à bon port nous aurions déposé notre cobaye dans un le canapé avant de lui offrir une banane fourrée avec des "smarties". Il ne restera plus qu’à filmer l’expérience pour la postérité et la science. 

Pour faire redescendre tranquillement notre petit patient sur le plancher des ruminants, nous lui diffuserions une sélection des meilleurs épisodes de Derrick, en version originale cela va de soi. \\

- Et s’il commence à nous faire un bad trip?\\
- Pas de problème, tu l’installes dans la douche et tu lui savonnes le dos.\\
- Tu es sur que cela suffira, ça me semble un peu léger comme médecine. \\
- Tu en connais beaucoup qui n’aimeraient pas ça ? Et bien pour lui c’est pareil, crois-moi si on leur installait un spa : ils passeraient leurs journées dedans à se faire brosser les omoplates. Eux, c’est sur ils y trouveraient largement leur compte, c’est juste que les visiteurs préféraient les voir s’élancer de cime en cime plutôt que de se passer la savonnette sur la fourrure. Le public en veut toujours pour son argent, ce qu’il veut c’est de l’image d’Epinal à la pelle et au diable les vrais besoins de ces petites bêtes. \\

Sans vouloir me jeter plus de roses à la figure que nécessaire, je dois bien avouer que j’étais très fier de notre plan. Les vapeurs d’alcool y étaient sans doute pour beaucoup. Je suis sur qu’en s’y replongeait avec la tête froide : le génie du plan apparaîtrait moins facilement à la première lecture. \\
La preuve ce matin même.\\

Je vous remets la scène pour ceux du fond qui auraient perdu le fil de l’histoire. Je suis donc sorti de la douche en trombe, dégoulinant d’eau de mousse avec juste une serviette nouée autour du ventre pour sauver mon reste de dignité et les apparences. Le téléphone dans une main et le produit à douche dans l’autre je m’évertue à composer le numéro de Yannick avec mes doigts mouillés. \\
Ce foutu produit à douche n’a pas été testé sur les animaux. Et dire que les crétins du marketing ont dû se congratuler en se donnant des grandes tapes dans le dos quand ils ont eu cette idée. Non testé sur les animaux, cela veut bien dire qu’ils ne garantissent absolument rien des effets secondaires qui pourraient subvenir si j’applique leur produit sur un lémurien. Le lémurien est un animal, je vous laisse faire le sophisme vous-même pour en arriver à la conclusion suivante : ce produit n’a pas été testé sur des lémuriens. Mais alors que se passerait-il si on se risquait à shampooiner notre cobaye : perdrait-il toute sa fourrure, serait-elle décolorée en blond vénitien, aurait-il des Anglaises sur tout son corps ? \\
Nous ne pouvions prendre de tels risques. C’est donc avec beaucoup de regret que nous n’avons pas pu apporter notre pierre à l’édifice scientifique. Adieu prix IgNobel, notre conscience prime sur notre besoin de reconnaissance. Merci à toi mon cher lecteur, qui comprend notre peine et notre décision.

