L’illumination, non que dis-je, le déclic m’est venu une nouvelle fois sous la douche. Archimède avait sa baignoire, j’ai ma cabine de douche. Chacun fait selon ses moyens.\\
Le flacon de produit que je tenais entre les mains indiquait fièrement la mention « non testé sur des animaux ». Je suis sorti en urgence, ruisselant encore de mousse pour me jeter sur mon téléphone et tout annuler. Avant d’aller plus loin dans les explications qui s’imposent, je vous propose de revenir en arrière et de commencer par le début. \\
Tout à commencer quand j’ai soumis l’idée de donner des hallucinogènes à une autruche. Enfin non, j’ai plutôt exprimé l’opinion que cette idée était particulièrement mauvaise et que je me félicitais de ne pas l’avoir eue. Pour de plus amples détails, je vous renvoie à la lecture de mon billet précédent. Ce qui devait arriver ne s’est pas fait attendre. J’ai reçu un message fort intéressant d’un employé de zoo qui m’a raconté avoir testé quelques expériences vaguement similaires par le passé. Par exemple, il s’est amusé à donner des laxatifs aux éléphants. Le résultat a été vraiment spectaculaire, éruptif même pour reprendre ses mots. Les pachydermes ont très, trop, bien réagis a cette médication. Peut-être s’est-il trompé dans les dosages, il m’a avoué ne pas maîtriser toutes les subtilités de la pharmacopée humaine appliquée aux animaux. Une autre fois, il a refilé des petites pilules bleues aux marmottes. L’enclos a été temporairement interdit aux visites scolaires durant presque deux semaines. Le temps que la libido de ces bestioles redescende à un minimum convenable pour un public non averti. 
En lisant le mail de ce monsieur, j’en suis naturellement arrivé à la conclusion suivante. Il fallait absolument que je le rencontre, afin bien sur, de le dissuader de recommencer une de ses nouvelles expériences. Souvenez-vous bien les petits enfants : la drogue c’est caca. 

Après quelques verres en sa compagnie, avec quelques tendant lourdement vers moult, nous avons finalement décidé d’épargner les autruches, les émeus et les casoars pour nous rabattre sur les lémuriens.\\
Les lémuriens ressemblent à des singes, même s’ils n’appartiennent pas du tout au même ordre. Ce sont des peluches avec de grands yeux exorbités et les gestes au ralenti. En sommes les candidats idéaux pour se voir prescrire du speed.

Avant que les défenseurs des animaux ne s’emparent de leurs fourches et de leurs flambeaux : sachez que nous avons pensé avant tout au bien être de ses petits junkys en fourrure. Nous ne pouvions les laisser dans la nature toute relative du zoo après avoir ingurgité une bonne dose de stupéfiant. Ce que nous avions prévu, c’était de les extraire de leurs cages pour les plonger dans un habitat plus rassurant. Imaginez-vous en plein milieu d’animaux sauvages dont l’immense majorité d'entre eux seraient des prédateurs sanguinaires. Chaque cri de hyène résonne dans le crâne du petit lémurien comme un glas funèbre. Le plus sécurisant aurait été de le placer dans un environnement plus feutré et édulcoré. Chez ma belle mère par exemple.

On a passé une bonne partie de la soirée à planifier l’opération dans ses moindres détails. Nous devions passer prendre notre patient quelques heures après la fermeture du parc zoologique. Le patient prendrait place dans le siège enfant à l’arrière de ma voiture, l’autoradio diffusera du Charles Aznavour en sourdine. Selon Yannick [c’est un prénom d’emprunt afin de garantir l’anonymat] les lémuriens ont un faible pour ce genre de musique. J’ignorais qu’il existait un top 50 chez les lémuriens. Si le grand Charles s’était produit dans l’enceinte du zoo : il aurait fait cage comble si vous me passez l’expression. On ferait bien attention à conduire suffisement lentement pour ne pas inquiéter notre passager mais pas trop pour ne pas attirer la maréchaussé. Une fois arrivé à bon port nous aurions déposé notre cobaye dans un le canapé avant de lui offrir une banane fourée avec des "smarties". 
