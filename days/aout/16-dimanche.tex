\section*{Dimanche 16}
\subsection*{Réveil matin}
Les matins n’ont jamais été mon fort, mais celui-ci encore moins que les autres. J’essaie de rassembler le reste de mes esprits. J’ai l’impression qu’on m’a remonté à l’envers et que mes yeux contemplent l’intérieur de mon crâne. Sinistre vision, je vous assure. 
Quelque part dans mon esprit encore embrumé une petite alarme retentit pour me dire qu’on m’appelle. Correction faite, on hurle mon prénom et le cri vient de la cuisine. Je me souviens de mon nom, cela veut au moins dire que la veille n’a pas été si terrible que ça. De mieux en mieux, je reconnais maintenant les lieux : je suis chez moi ! 

Nouvel appel d’urgence. Toujours sur le même ton et la même intensité sonore. Mes quelques neurones survivants crient grâces. L’apocalypse oui, mais en silence s’il vous plaît. Je me risque à une réponse qui manque de me faire sortir le cerveau par les oreilles et je me mets courageusement en route vers l’origine des appels. Le couloir tangue dangereusement durant toute la traversée, et je finis par reprendre équilibre en me posant façon atterrissage forcé moitié sur le tabouret et moitié sur la table de la cuisine. N’ayez crainte gente dame, me voilà !

Mon épouse me fait face. Je tente un sourire charmeur pour désamorcer la situation dont j’ignore encore tous les tenants et aboutissants. Ses yeux passent de la noisette au noir assassin, c’est mauvais signe. Il est minuit vingt à l’horloge de l’apocalypse, rien ne va plus les jeux sont mal faits. J’ignore encore de quoi on m’accuse, mais je ne risque pas de m’en tirer avec un simple rictus.
Elle s’est attaché les cheveux en chignon à la mode samouraï elle pointe un doigt bien aiguisé en direction du bout de la table. 
- C’est quoi ça !
Les yeux parcourent tant bien que mal le chemin vers l’emplacement pointé par l’index vengeur et finissent leur course sur ça. Ça c’est un lémurien avec de grands yeux apeurés et les deux bras levés vers le ciel comme les voleurs pris en flagrant délit par la police.
- Oh c’est pas grave, je lui dis qu’il avait le droit.
- Le droit de quoi ?
- De prendre mes BN, faut bien partager quand même.
- Mais c’est quoi ça !
Ton homicide de la réplique ne laisse pas beaucoup de doute, je me suis fourvoyé sur ma tactique de réponse. Surtout garder son calme et reprendre la main.
- D’abord c’est pas ça, mais c’est qui. Et pour tout te dire c’est Roger. Roger, je te présente ma chère et tendre. Hier soir, on a fini un peu tard, et du coup j’ai proposé à Roger de dormir à la maison.
Le regard de ma chère et tendre me traverse l’occiput de ses prunelles noires.

