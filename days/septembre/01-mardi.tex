\section*{Mardi 1}
\subsection{cidre}

On est mardi et en ce jour de rentrée, j’ai envie de prendre tout le monde à rebours de vous évoquer un souvenir de vacances. J’aurais pu faire preuve d’un peu de compassion et penser à vous qui êtes enseveli sous les piles d’attestations scolaires à signer et les montagnes de livres à recouvrir. J’aurais pu, oui, mais je sais qu’au plus profond de vous : vous n’aspirez qu’à une chose prendre virtuellement le large en écoutant une de mes histoires.
Peut-être que je me monte un peu la bourriche et que je suis loin du compte mais l’avantage du livre c’est que je ne vous laisse jamais voix au chapitre. Je reste donc sourd à vos protestations et je reprends la barre de mon livre. \\

Comme bien souvent l’été quand le soleil brille un peu trop et que les plages bretonnes sont prises d’assaut par des hordes de touristes parisiens ; je m’en vais voir ailleurs si j’y suis. Sans grande surprise, j’y suis rarement. Cette année l’ailleurs c’est trouvé être en Normandie. C’est une région magnifique, pas autant que la Bretagne bien sur, mais il n’en reste que c’est un recoin qui mérite bien qu’on y fasse le détour. Cette terre chantée pas Stone et Charden est pleine de bonnes surprises pour celui qui prend la peine de s’y attarder.\\

Je me promenais dans un petit havre tranquille, et afin qu’il le reste : j’en tairais le nom, un jour de marché. La gastronomie normande partage un certain goût des bonnes choses et une haine farouche de la diététique avec la cuisine de mon terroir. La générosité en toute chose et surtout dans l’usage de la crème. Je flânais entre les étals des pâtissiers en me demandant pourquoi je n’avais pas eu la présence d’esprit de prendre un caddy. Soudain mon regard fut attiré par un marchand de cidre. Du cidre oui mais du cidre breton de Normandie. N’ayant aucune envie de raviver une querelle aussi vieille que la propriété du Mont Saint Michel, je ne me lancerais pas dans une comparaison entre les différents cidres. Une inconnue demeure : comment fait-on du cidre breton en Normandie ? Sans chauvinisme aucun l’inverse m’aurait tout autant étonné. 
Je m’en suis donc allé quérir les réponses à la source, directement chez le producteur.\\

Yannick est à l’origine un brestois pur beurre, ses parents n’ont jamais quitté la rade des yeux et n’y ont sans doute jamais songé un instant. Lui c’est une tout autre affaire. Il rencontre sa future épouse lors de ses études supérieures à Rennes il prend le large pour la Normandie. La belle est normande, et c’est à regrets qu’il quitte la terre de ses ancêtres. Le bon côté de la chose, c’est qu’au moins là-bas il y a la mer. Quelques années après s’être installé comme comptable, sa femme et lui hérite d’une cidrerie avec une pommeraie attenante. Yannick décide de conserver le bien et se lance dans la cidrerie artisanale sur son temps libre.\\
Vous imaginez bien que ce ne sont pas les producteurs de cidre qui manquent dans le coin. La concurrence est rude surtout pour un amateur qui se pointe sur un marché déjà bien rempli. C’est là que Yannick à son coup de génie. Enfin c’est un point de vue, je vous en laisserais juge mais ayez au moins l’honnêteté d’avouer que sa démarche est pour le moins originale. Breton jusqu’à la substantielle moelle il ne pouvait faire que cidre breton. Ma grand-mère m’aurait renié si j’avais fait autrement, m’a-t-il avoué. Certes mais comment fait-on, concrètement avec des pommes normandes dans une cidrerie normande avec des foudres en chênes normands ? Et je m’arrête là, mais vous avez certainement compris l’idée.\\

Pour Yannick on est breton dans l’âme. Un breton à Oulan-Bator ou à Brazzaville, ça reste un breton. Un breton en Papouasie même avec un os dans le nez et un pagne en feuilles et bien ça reste un breton quand même. Devrais-je préciser qu’au moment où il a prononcé cette phrase le niveau de la bouteille de poirée avait dangereusement baissé. Si le cidre est breton dans l’âme c’est le plus important que son lieu de fabrication. Personnellement, je ne suis pas certains que ces messieurs de la répression des fraudes partagent ce point de vue, mais passons.\\
Le tout est de donner de la bretagnosité (l’adjectif est de lui, et le niveau de poirée frôlait la marée basse) au cidre. Pour ce faire, tout doit se jouer dès les premiers stades. Les pommes sont élevées (le niveau de poirée était tombé au niveau du plancher des taupes) dès leur plus jeune âge au son de la cornemuse.\\ Je leur passe du Tri Yann,du Alan Stivell et parfois même du Nolwenn Leroy. Je leur parle en breton bien sur, je leur parle des cotes de granit, des marins au long court et des mégalithes. Quand le bouchon saute ça pétille en breton, ça fleure bon le kouign aman qui sort du fournil et les embruns.\\

De la part d’un type qui aime tellement son boulot qu’il a été jusqu’à baptiser ses pommes une à une, le moins qu’on puisse dire c’est que son cidre est fait avec amour. Si à l’occasion vous passez dans le coin, n’hésitez pas à en goûter une bolée ou deux, ça vaut le détour.
