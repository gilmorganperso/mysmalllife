\section*{Samedi 5}
\subsection*{Annie Cordy}
Hier soir, la boite à conneries était allumée sur BFM, Claire était dans le champ de tir de cette affreuse machine à désinformer quant à moi j’étais bien planqué derrière un livre à attendre l’extinction. Soudain une voix me tire de ma lecture : Annie Cordy est morte.\\
Je lève un regard mauvais, un œil suspicieux et l’autre passablement plus noir qu’a son ordinaire. BFM est connu pour cracher plus de fausses informations à la minute qu’une bonne mitrailleuse bien huiler peut envoyer de balles sur une foule de civil. Alors première réaction, sur le vif, ce n’est ni drôle ni à faire. Le Gorafi aurait-il réussi à percer jusqu’aux chaînes cathodiques ? Faudrait qu’on me tienne au goût du jour, ce n’est pas parce que je refuse de regarder la télévision qu’il faut me cacher ce genre d’information.
Un rapide coup d’œil sur les autres canaux d’informations me confirme cette triste nouvelle. Annie Cordy est morte.\\

Alors non ce n’est pas possible, pas elle. Annie Cordy est immortelle, elle fait partie des meubles et du paysage. Annie c’est comme les vaches en Normandie ou le bal du 14 juillet, c’est immuable et ancré. Le journaliste en question aurait annoncé de la neige en juillet, j’aurais pu faire semblant d’y croire. Avec le réchauffement climatique, le golf stream qui se fait la malle ou je ne sais quoi d’autre. Ça j’aurais pu y croire, mais Annie Cordy non. \\
Je n’ai pas envie de leur demander de me montrer le corps, j’ai trop peur qu’ils m’entendent de l’autre coté de l’écran et qu’ils versent un peu plus dans le voyeurisme morbide. Charognards que vous êtes, vous étiez vraiment obligé de nous balancer l’info comme ça. Juste un petit entrefilet en bas d’écran entre les cours de la bourse et les scores du foot. Vous ne respectez donc rien, et sachez que venant de ma part, l’insulte vaut son pesant de fiel.\\

Annie Cordy, n’est pas morte. Je le sais, car ce n’est tout simplement pas possible. Pas elle. Elle avait tout simplement renoncé de vieillir. Même à l’age d’être grand-mère et même arrière grand-mère elle faisait preuve d’une incroyable énergie et d’un sourire à toute épreuve. Ah ce sourire, parlons en donc un peu, c’est un sourire comme il n’en existe pas assez, hélas. Rien à voir avec le sourire plein de fausses dents et de fausses promesses d’un homme politique. Rien à voir avec le sourire carnassier d’un requin commercial qui s’apprête à dépecer une cohorte de salariés pour quelques stocks options. Rien à voir avec le sourire béa d’un exilé des paradis artificiels. Un vrai sourire, franc et communicatif. J’espère qu’elle est partie avec ce sourire aux lèvres. \\

Annie Cordy, c’est une institution. Ce celui qui n’a jamais chanté une de ses chansons se jette la première pierre. On ne peut pas avoir interprété autant de chansons qui ont traversé avec autant de facilité les années sans faire preuve d’un indéniable talent. Moi-même je dois avouer que je me suis souvent retrouvé dans des soirées bien arrosées à brailler à tue-tête Tatayoyo ou La bonne du curé avec une bonne chorale de petits chanteurs à la gueule de bois. Je pense que tous ceux qui ont levé leurs verres, en s’en reversant la moitié sur le crane, en reprenant les refrains de la grande Annie partagent un peu de ma peine.\\

Annie Cordy qui passe l’arme à gauche, c’est impensable. Avoir autant d’énergie et décider de s’arrêter d’un coup d’un seul, ça n’a aucun sens. Il faut croire qu’il y a dû avoir une méprise quelque part, une erreur d’aiguillage sur le karrig an ankou, la faucheuse qui s’est trompé d’une porte.\\

Avec son départ, c’est une bonne partie de ma jeunesse qui s’enfuie avec la chanteuse de « Cigarette, whisky et p’tites pépées ». Le vous laisse le dernier couplet comme épitaphe. \\


« Bye bye la vie, … y a pu rien à regretter. » 

