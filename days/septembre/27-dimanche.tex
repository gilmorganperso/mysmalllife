\section*{Dimanche 27}
\subsection*{Tartine}
Il n’y en aurait pas de quoi en faire toute une tartine, et bien peut-être que si au contraire.\\

Laissez-moi vous faire une petite mise en situation avant de me lancer dans le gros de ma démonstration. Nous sommes pile à l’heure de déjeuner, pour autant que cette heure puisse varier d’un individu à un autre, je suis la référence de cette histoire. Après tout c’est moi qui vous narre ce qu’il s’est passé, a priori je suis le mieux placé pour placer les références. Il est l’heure de se substanter, étant donné que je n’ai pas de ruban bleu, ni de col tricolore, j’ai pris la liberté de reprendre à mon compte une recette éprouvée par les siècles. La fameuse tartine de pâté, qui se décline aussi en tartine de rillettes, ou tartine de Nutella pour les amateurs d’huile de palme. Le principe est simple on prend une tranche de pain et on applique dessus le condiment qui convient. Jusque-là, rien de bien compliqué, poursuivons.\\
J’ouvre le sac de pain de mie, je m’empare d’un geste assuré d’une tranche que je catapulte dans la foulée sur la table avant d’attraper le pâté de foie. Et soudain, je m’arrête, pris d’un doute. Je contemple d’un air soupçonneux la tranche de pain.\\

L’ai-je mise du bon côté. Si on fait abstraction de la tranche, on peut simplifier en disant qu’une tranche de pain à deux cotés. Toujours sur cette logique, il devrait normalement y avoir un coté haut et un côté bas. Mon raisonnement semble logique, à ceci prêt qu’il se pose ensuite la deuxième question. Comment différencier les deux faces l’une de l’autre ?\\

La loi de Murphy et ses corollaires n’offrent aucune explication sur ce point précis. Une tartine tombera toujours du côté beurré, c’est un fait. Une tartine possède naturellement un haut (le côté beurré ou tartiné) et un bas l’autre coté. Le premier des imbéciles (qui n’aura pas le loisir de s’asseoir à ma table) saura prendre une tartine du bon côté. Cette convention est en vigueur dans tous les pays civilisés qui pratiquent la tartine et accessoirement aussi le hooliganisme et la peine de mort. Je m’égare, pardon.\\

Avant d’être une tartine, une tranche de pain n’est qu’une tranche de pain. C’est-à-dire qu’avant la phase finale de sa recette, peut-on vraiment dire qu’elle à un haut et un bas. Un coté table et un coté rillettes si vous préférer cette analogie. C’est une question philosophique qui mériterait sans doute qu’on l’approfondisse. Mais je tiens à rappeler que je ne suis ni philosophe ni grec et que j’ai mieux à faire que passer mes journées en toge à méditer sur le sens de l’univers, de la vie et du reste.\\

Ne pourrait-on pas demander au fabriquant de pain de mie de mettre un signe distinctif sur une des faces des tranches. Sans aller jusqu’à imposer une flèche avec un « this side up », une simple croix dans un des coins serait amplement suffisant. A condition bien évidement de fournir une notice explicative pour préciser si le symbole désigne quel coté.\\
On pourrait aussi faire plancher la communauté européenne pour imposer une législation sur le marquage des tranches de pain et autre biscottes. Pour une fois que ces messieurs des commissions pourraient faire quelque chose d'utile de leur journée ce serait domage de ne pas utiliser nos impôts à bon essient. Enfin, comme toujours, je dis ça, je dis rien.\\

En guise d’épilogue, j’ai tartiné les deux cotés. Juste histoire de ne pas me tromper.
