En relisant en diagonale les pages de mon tapuscrit, je me suis vite rendu compte qu’il manquait quelqu’un. A aucun moment au fil des pages, je n’ai mentionné celle qui rend tout ceci possible : ma muse. Il va sans dire que si je continue à l’omettre de mes diatribes elle risque de m’en vouloir et par voie de conséquence se venger. D’autant plus que je ferais preuve d’une immense goujaterie doublée d’une injustice flagrante en ne lui rendant pas l’hommage qu’elle mérite.
Ma chère muse, ce texte est pour toi.
Toutefois mon cher hypothétique lecteur, laisses-moi te clarifier une ou deux choses avant qu’il n’y aie de méprise en la demeure. Quand je parle de ma muse, il ne s’agit pas de ma chère et tendre, celle qui partage mon cassoulet et mon quotidien. Celle qui subit mes soliloques sans trop broncher et qui continue de surveiller ses casseroles pendant que je peste contre le présentateur du journal télévisé en cherchant la télécommande à quatre pattes.
Non, quand je dis muse, je pense muse au sens premier du terme, au sens de Μοῦσα. Que les non hellénistes reposent leurs fourches, je m’en vais de ce pas vous expliquer de quoi il en retourne exactement. Ce barbarisme désigne les muses dans son acceptation mythologique, c’est-à-dire les neuf filles de Zeus et Mnémosyne. Non mais l’autre comment il balance ma vie de famille à tout le monde. Non mais va-z-y te gênes surtout pas. Enfin neuf, c’est vite dis. Hegel n’en comptait que cinq, à croire que quatre d’entre elles, une petite moitié, avait disparu au cours de l’histoire. Je pense que c’est un peu plus compliqué, et que si on ne pourrait parler au sens propre de disparition on pourrait évoquer plutôt une transformation. Force est de constater que la poésie lyrique ne déchaîne plus les passions comme avant, alors il a fallut se recycler. Certaines muses ont définitivement pris leur retraite, d’autres ont pris la relève, beaucoup de muses ont été embauchées. A une époque, l’antiquité ou les lettrés étaient rares c’était plutôt simple, de nos jours avec l’école obligatoire l’analphabétisme à reculer en flèche (en théorie du moins). Couplons ce phénomène avec la démocratisation des réseaux sociaux qui permettent à tous de s’exprimer publiquement. Ouais enfin surtout ceux qui n’ont rien à dire, suivez mon regard.
Sachez qu’on ne choisit pas sa muse c’est elle qui vous choisi. Ne l’écoutez pas, j’ai pas eu mon mot à dire. C’étais lui ou la porte, je me demande encore si j’ai fais le bon choix.
