\section*{Mardi 8}
\subsection*{Pastis}

Le pastis.\\
Je devais écrire un article sur le pastis et pourtant je m’étais promis de ne pas parler de politique. Oui, car j’ai appris sur le tard que le pastis est politique. Pour ma décharge, je n’en bois jamais et je n’aime pas l’anis. Il est de notoriété publique que mes penchants naturels m’entraînent vers le rhum.\\

Pastis signifie bouillie en vieux provençal alors avant d’aller plus loin entendons-nous bien sur la définition du liquide. Le pastis est un liquide blanc jaune, composé d’une base d’alcool agrémentée d’anis, de réglisse, de fenouil et de coriandre. Pas étonnant au vu de la recette que les gens du sud aient décidés de baptiser cette soupe : pastis. En fait le pastaga c’est un peu la version alcoolisée de la bouillabaisse.\\
Pour la théorie, je suis certain qu’il existe d’excellent livre sur le sujet, mais pour autant rien ne vaux l’avis éclairé d’un professionnel. Je me suis rendu dans un troquet tenu par un éminent spécialiste. Le tôlier est marseillais comme ce n’est pas possible. Son accent est à tronçonner à la disqueuse diamantée. C’est tout Pagnol et les cigales qui parlent dans sa voix. On aperçoit les calanques à travers chaque voyelle qu’il prononce. Même sa démarche n’est pas sans rappeler celle des pétanquistes s’en allant rejoindre le cercle sur la place de l’église. Je doute qu’il soit possible de retranscrire avec justice ce phrasé unique, aussi je ne m’y risquerai même pas.\\

Il m’a jeté un drôle d’air lorsque je lui ai commandé un pastis, et un air encore plus désemparé quand je lui ai annoncé que je n’avais pas l’intention de le boire.\\

- Pourquoi commander un verre si c’est juste pour le regarder, c’est pas avec les yeux que tu vas vider le ballon. La plupart de mes clients sont capable d’engloutir godet sur godet sans même y jeter un œil et toi tu cherches à faire l’inverse.\\
- Je cherche juste l’inspiration pour parler du pastis.\\
- T’en as de bien bonnes. Toi qui me fais souvent la morale car monsieur ne veut pas parler de politique et qu’est-ce que j’entends : tu veux t’étaler sur le pastis.\\
- C’est pas politique le pastis. Enfin pour ce que j’en sais.\\
- Hé bien faut croire que t’en sais autant que t’en bois, c’est-à-dire pas goutte. Laisse-moi te dire, il y en a qui te dise que le pastaga c’est une religion que le petit jaune c’est sacré. C’est de la foutaise, ces gens-là ne méritent même pas de s’approcher de mon comptoir. Écoute-moi bien, le pastis c’est politique mais attention y a pastis et pastis. Et faut pas tout confondre, ça n’a rien à voir.\\
Tu as le pastis parisien, le truc que tu avales vite fait en penchant la tête en arrière. Il est tout juste tiède mais c’est pas grave, tu le bois, car on t’a demandé de le faire et que c’est juste pour faire comme tout le monde. C’est un pastis sans conviction, ça devrait même pas compter.\\
Y a pire encore, le pastis des hauts quartiers. C’est tout juste si on te le sert pas dans une flûte en cristal avec un napperon en dessous. A ce niveau, ça rime plus à rien. Payer une bouteille la moitié d’un smic, c’est presqu’autant du vol de la bêtise. C’est même pas de l’indécence, car je trouve qu’il y a plus beau à montrer son cul qu’à se vautrer là-dedans.
Dans les trucs qui me font monter la bile, je te passe aussi le pastis des bobos. Fallait l’inventer et ils l’ont fait. Ils ont collé un trèfle vert [NDLA : je n’ai pas la référence] sur la bouteille et hop, c’est bio, c’est tout bon pour la planète. Ca les empêche pas de le noyer dans de l’eau minérale qu’a traversé la moitié du globe sur un tanker rouillé qui pue tellement le mazout que même les mouettes s’en approchent pas.\\
Y a les fanfarons qui se montent du col au comptoir comme des oies pour le gavage. Ça veut le boire dans le beau verre bien, versé à la dosette avec mesure, avec la carafe qui va bien et l’eau jusqu’au trait. C’est tout juste s’ils te demandent pas une pipette pour le dosage. Bandes de mulots, c’est pas de la chimie, c’est de l’art de le servir le pastis. Ce genre de gars, ils font tous juste sur les traits et puis ils se plantent bien pour que tout le monde les voit bien. C’te cagoles, que j’te balancerais ça à la rue à coup de balais.\\
Et puis enfin, il y a le vrai buveur de pastis. Lui il a rien à prouver à personne. Son pastis, il le boit pour lui. Et tu lui feras pas à l’envers, tu vas pas lui fourguer n’importe quoi. Et tu t’amuses pas à lui couper. L’eau il la verse comme un chef sans qu’on lui tienne la main. Lui il vient au bar pour boire son pastis, car il sait qu’il l’a bien mérité. Il vient communier avec les siens. Les autres aux bars, ils disent rien non plus, mais ils se comprennent, ils se savent. Le pastis il est comme il faut, l’eau est à la bonne température et il y en a juste ce qu’il faut.\\
Toi, tu as un jolly roger planté dans ta tête de mule, c’est pour ça que tu dois que du rhum. Tu auras beau essayer de le cacher, ça se voit comme le nez au milieu de la trogne. Depuis le temps que je te chante les louanges du pastis, t’as même pas pensé une fois à tremper tes lèvres de dedans. T’es vraiment irrécupérable.\\
Bon je te prends ton pastis et je te sers ton tafia car sinon tu vas encore boire tout seul comme un couillon au coin du zinc. Ah c’te figure.\\
